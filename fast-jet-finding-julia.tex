% This paper is Copyright © Graeme Andrew Stewart, Philippe Gras,
% Sanmay Ganguly, Sattwamo Ghosh and Atell Krasnopolski, 2025
% Licensed under Creative Commons Attribution 4.0 International (CC BY 4.0), see LICENSE

\documentclass{webofc}
\usepackage{graphicx} % Required for inserting images
\graphicspath{{figures/}}
\usepackage[varg]{txfonts}
\usepackage[utf8]{inputenc}

% Workaround for arXiv, start with "finalizecache" then switch to "frozencache"
\usepackage[finalizecache,cachedir=.]{minted}
% \usepackage[frozencache,cachedir=.]{minted}

\usepackage{natbib}
\usepackage{hyperref}
\usepackage{enumitem}

% For draft version
\usepackage{lineno}
\linenumbers

\newcommand{\kt}{${k}_\text{T}$}
\newcommand{\akt}{anti-${k}_\text{T}$}
\newcommand{\Akt}{Anti-${k}_\text{T}$}
\newcommand{\JR}{\texttt{JetReconstruction.jl}}
\newcommand{\ee}{$e^+e^-$}

\title{Fast Jet Finding in Julia}

\author{\firstname{Graeme Andrew} \lastname{Stewart}\inst{1}\fnsep\thanks{\email{graeme.andrew.stewart@cern.ch}} \and
\firstname{Sanmay} \lastname{Ganguly}\inst{2} \and
\firstname{Sattwamo} \lastname{Ghosh}\inst{2} \and
\firstname{Philippe} \lastname{Gras}\inst{2}\fnsep \and
\firstname{Atell} \lastname{Krasnopolski}\inst{3}
% etc.
}

\institute{CERN, Esplanade des Particules 1, Geneva, Switzerland
\and
Indian Institute of Technology, Kanpur, India
\and 
IRFU, CEA, Université Paris-Saclay, Gif-sur-Yvette, France
\and
Julius-Maximilians-Universität Würzburg, Würzburg, Germany
}

\abstract{%
Jet reconstruction remains a critical task in the analysis of data from HEP
colliders. We describe in this paper a new, highly performant, Julia package for
jet reconstruction, \JR, which integrates into the growing
ecosystem of Julia packages for HEP. With this package users can run sequential
reconstruction algorithms for jets. In particular, for LHC events, the \Akt,
Cambridge/Aachen and Inclusive algorithms can be used. For FCCee studies the
use of alternative algorithms such as the Generalised \kt\ and Durham are also
supported.

The full reconstruction history is made available, allowing inclusive and
exclusive jets to be retrieved. The package also provides the means to visualise
the reconstruction.

The implementation of the package in Julia is discussed, with an emphasis on the
features of the language that allow for an easy to work with, ergonomic, code
implementation, that achieves high-performance. Julia's ecosystem offers the
possibility to vectorise code, using single-instruction-multiple-data
processing, in way that is transparent for the developer and more flexible than
optimization done via C and C++ compilers. Thanks to this feature, the
performance of \JR\ is better than the current Fastjet C++
implementation for jet clustering of $pp$ events produced at the LHC
and for typical events for FCCee.

Finally, an example of an FCCee analysis using \JR\ is shown.}

\begin{document}

\maketitle

\section{Introduction}
\label{sec:introduction}

High energy physics (HEP) software is inherently
multi-lingual~\cite{pivarski2022}. Across the field actively used codes exist in
Fortran, C++, Python, Go, Java, Javascript -- and this is certainly an
incomplete list, if all corners were to be explored. However, for code that is
used in mainline HEP workflows two languages have dominated in the last few
decades: C++ and Python. C++ saw major adoption in the BaBar experiment at SLAC
and then the experiments at the Large Hadron Collider. Python has also grown
enormously in popularity in recent years for the LHC experiments, indeed
becoming and almost ubiquitous across the field. These languages have different
strengths, with C++ excelling at runtime performance, used heavily for
simulation and reconstruction; and Python shining in the areas of rapid
turnaround, prototyping and steering, being then particularly strong in the
analysis domain.

The current status of C++ and Python, with these languages having rather
different paradigms, leads to friction and potentially awkward interfaces. Code
first developed in Python may not run efficiently at scale, leading to
inefficient use of computing resources; that may necessitate a rewrite in C++.
The current generation of physicists is generally far more comfortable in Python
and there is a loss of skills in C++.

An alternative option, which is attracting increasing interest, is to use a
language that can bring the runtime advantages of C++, but the ergonomic ease of
Python. The \emph{Julia Programming
Language}~\cite{bib:julia_freshapproach,10.1145/3276490} was designed
specifically to do this efficiently and effectively, and has been adopted by
many scientific communities~\cite{perkel-julia-science}. In HEP explorations of
Julia have been promising~\cite{Stanitzki:2020bnx,eschle2023potential}.
In particular, a recent comparison of Julia, Python and C++ for the task of
\emph{sequential jet finding}~\cite{polyglot-jets-chep23} fond that Julia
performed as well as, or better, than C++, with improved code ergonomics. 

In this paper we report on the developments that have happened in the Julia code
presented in~\cite{polyglot-jets-chep23}, in particular the improvements that
have resulted in the recent release of the production Julia package,
\JR~\cite{jetreconstruction-jl-github}. The package has been made more
accessible to users, with comprehensive documentation and consistent interfaces.
New algorithms have been introduced, specifically targeting jet reconstruction at
\ee\ experiments, including reading data from Key4HEP's EDM4hep data format
files. Support for jet substructure analysis at $pp$ colliders has been
introduced. We also give the latest benchmarking results, that continue to
demonstrate faster runtimes than Fastjet~\cite{Cacciari:2011ma} for almost all
parameters.

\section{Production Release of Julia Jet Reconstruction}
\label{sec:prodrel}

Details of the algorithms and strategies used in the Julia version of jet
finding have been described in~\cite{polyglot-jets-chep23}. However, at the time
of that publication, due to the different development history of how the two
strategies were implemented, the return values from the reconstruction were
different. In the \texttt{N2Plain} case an implicit $p_T$ cut, selecting
inclusive jets, was made; whereas in the \texttt{N2Tiled} strategy a dedicated
object, called a \texttt{ClusterSequence} was returned. The advantage of the
latter (which was inspired by Fastjet) is that it stores the entire history of
the reconstruction process. Therefore it is much more useful for subsequent
processing and analysis of the jet reconstruction. Therefore, for the production
release this was unified to returning a \texttt{ClusterSequence}.

Once \texttt{ClusterSequence} was identified as the common return type, this
allowed the development of the other core jet selection, viz.\ \emph{inclusive
jets}. An interface was added where the same method can be used to make a
selection on either the number of final jets, or on the maximum value of the
metric distance ($d_{ij}$) -- this takes advantage of the fact that in Julia
method parameters can be named, providing a clearer interface than type based
method selection in C++.

In addition to the jet selection, another algorithm for $pp$ jet reconstruction
was added, the generalised \kt algorithm. This algorithm uses as a momentum
metric $k^{2p}_\text{T}$ where the power value $p$ is arbitrary (and for
specific integer powers of $-1$, $0$ and $1$, maps to the well known \akt,
Cambridge/Aachen, and inclusive \kt\ algorithms, respectively).

Another adaption to \texttt{ClusterSequence} was to interface the pre-existing
jet visualisation support in the package. This support comes from the Julia
visualisation package \emph{Makie}~\cite{Danisch2021}. As Makie is a heavy
dependency, we take advantage of the \emph{extensions} feature of the Julia
packaging system, where the visualisation extensions to \JR\ are only loaded if
\texttt{Makie} already exists in the current user environment. An example of the
output from the visualisation extension is shown in Figure \ref{fig:jetvisplot}.
Taking advantage of the fact that all reconstruction steps are captured by the
\texttt{ClusterSequence} a new visualisation option was added, which animates
the reconstruction process~\cite{jetrecoAnimationCHEP2024}.

\begin{figure}[h]
    \begin{center}
        % TODO: improve the quality of this plot. %
        \includegraphics[width=0.8\linewidth]{jetvis-5-compact.pdf}
        \caption{Visualisation of a typical $pp$ collision jet reconstruction, using \Akt, in the $y-\phi$ (rapidity, azimuthal angle) plane. The height of each bar indicates the original cluster energy and the colour represents the final jet clustering, i.e., all clusters with the same colour are clustered together. In this example $R=2$, to make the final clusterings clearer.}
        \label{fig:jetvisplot}
    \end{center}
\end{figure}

The difference between the two core strategies is that \texttt{N2Tiled} scales
much better to higher initial cluster densities, very much found for LHC $pp$
events. However, there is an overhead for this tiling, which makes the
\texttt{N2Plain} strategy better at low cluster densities. It is highly
desirable that the user would not have to manually select a strategy, so
heuristic performance scan was made, indicating that the performance of each
strategy is about the same for 80 input clusters. Therefore a third strategy,
\texttt{Best} was introduced, which selects \texttt{N2Plain} for 80 clusters or
less, otherwise \texttt{N2Tiled}. this can be further extended in the future, if
specialist strategies for very high densities are added, e.g., to deal
specifically with heavy ion type events.

Before a useful release of the software could be made, documentation for the
package was written. This was done using the standard Julia documentation
support package, \texttt{Documenter.jl}~\cite{documenter-jl}, which has the
great advantage of using the inline code docstrings to document methods. This
allowed us to use AI (Github Co-pilot) to bootstrap docstrings from the code
itself. Documentation was then published onto the JuliaHEP organisation's
GitHib Pages website~\cite{jetreco-docs}.

With all of this refactoring done, and with an enhanced suite of tests added,
the first public release of \JR\, v0.3.0, was made in June 2024. The package was
added to the Julia public registry, making installing it for any user as simple
as \texttt{add JetReconstruction} from the standard Julia package interface.

\section{Support for \ee\ algorithms}
\label{sec:ee}

To add support for the reconstruction of jets in \ee\ events, some different
algorithms are needed. The essential idea of sequential jet reconstruction
remains the same: calculation of a distance metric between all clusters, then
merging the clusters with the lowest metric. However, for \ee\ events it is
preferable to reconstruct in geometric space, $(\theta, \phi)$ instead of $(y,
\phi)$. This is because experiments usually operate at the production threshold
of the processes of most interest, so objects of interest are less boosted than
at the LHC.

There are two main algorithms of interest: the Durham Algorithm and the Generalised \kt\ for \ee.

\subsection{Durham Algorithm}
\label{sec:durham}

For the Durham Algorithm the metric distance between jets $i$ and $j$ is defined as:

$$
d_{ij} = 2 \text{min}(E_i^2, E_j^2) (1 - \cos \theta_{ij})
$$

where $E_{i}$ is the cluster energy and $\theta_{ij}$ is the angular separation between $i$ and $j$.

The reconstruction implementation in \JR\ is then called with

\begin{minted}{julia}
    cs = jet_reconstruct(particles; algorithm=Durham)
\end{minted}

No additional parameters are required. In principle the Durham algorithm
proceeds until all clusters are merged to a single jet, but actual analysis will
utilise an exclusive jet cut, for the number of jets of interest.

\subsection{Generalised \kt\ for \ee}
\label{sec:getktee}

For the Generalised \kt\ for \ee\ algorithm, the distance metrics are:

$$
d_{ij} = \text{min}(E_i^{2p}, E_j^{2p}) \frac{1 - \cos \theta_{ij}}{1 - \cos R}
$$
$$
d_{iB} = E_i^{2p}
$$

For an power value $p$ and a radius parameter $R$. This means jets are finalised
when no clusters are found within angular distance $R$ of when $R<\pi$ (this is
very similar to the behaviour of the $pp$ algorithms). In \JR\ we follow the
Fastjet prescription that for $R>\pi$ the denominator is replaced by $3+\cos
R$~\cite{fastjetmanual}.

In the case when $p=1$ and $\pi < R < 3\pi$ the clustering sequence is identical
to the Durham Algorithm (save for a normalisation factor of 2).

\subsection{Implementation Details}
\label{eeimplementation}

For an optimal implementation of the \ee\ algorithms we introduce a new Julia
structure to represent an jet to be reconstructed in $(\theta, \phi)$ space, an \texttt{EEjet}.

\begin{minted}{julia}
mutable struct EEjet <: FourMomentum
    px::Float64
    py::Float64
    pz::Float64
    E::Float64
    _p2::Float64
    _inv_p::Float64
    _cluster_hist_index::Int
end
\end{minted}

This structure mainly differs from the \texttt{PseudoJet} used for $pp$
reconstruction in that the cached values are optimised for the different
reconstruction scheme. Both \texttt{EEjet} and \texttt{PseudoJet} are subtypes
of the abstract type \texttt{FourMomentum}, which allows us to parameterise the
\texttt{ClusterSequence} on the jet type, thus benefiting from a type specific
implementation at runtime, with generic code for the \texttt{ClusterSequence}, viz.

\begin{minted}{julia}
struct ClusterSequence{T <: FourMomentum}
    ...
    jets::Vector{T}
    ...
end
\end{minted}

\section{$pp$ and \ee Performance}
\label{sec:performance}

We have benchmarked the performance of \JR\ v0.4.3 against Fastjet v3.4.3\footnote{The benchmark machine used was an AMD Ryzen 7, 5700G 3.8GHz (8 cores, plus HT), 32GB RAM, running AlmaLinux 9.4. Julia v1.11.1 was used for \JR\ and Fastjet was compiled with gcc 11.4.1 using \texttt{-O2}. Benchmark runs are repeated 32 times and are stable to 1\%.}.



\section{Substructure and Taggers}
\label{sec:sstag}

\section{FCCee Jets and EDM4hep}
\label{sec:fccee}

\section{Conclusions}
\label{sec:conclusions}

\sloppy
\raggedright
% \clearpage
\bibliography{fast-jet-finding-julia}

\end{document}
